% Author: Pakkapon Phongthawee

% !TEX program = xelatex

\documentclass[xcolor=dvipsnames, xetex,serif]{beamer}
%\documentclass[handout,xetex,serif]{beamer} %ใช้บรรทัดนี้สำหรับปริ้นเอกสาร
\usepackage{color,amsmath,graphics,graphicx}
\usepackage{epsfig,amsfonts,graphics}
\usepackage{mathrsfs,hyperref}
\usepackage{subcaption,float,framed,algorithm2e,hyperref}
%===============================================
\usepackage{fontspec,xltxtra,xunicode}
\defaultfontfeatures{Scale=1.23}
\XeTeXlinebreaklocale “th_TH” % สำหรับตัดคำ
\setmainfont[Scale=1.23]{THSarabunNew}
% 1.23 เท่าคือจาก 12 pt บน LaTeX ให้เท่ากับ 16pt บน Word
%=====================================================
%\usepackage{pgfpages} %ใช้บรรทัดนี้สำหรับปริ้นเอกสาร
%\pgfpagesuselayout{4 on 1}[a4paper,border shrink=5mm,landscape]
%\pgfpagesuselayout{2 on 1}[a4paper,border shrink=5mm]
%ใช้บรรทัดนี้สำหรับปริ้นเอกสาร

%%%%%%%%%%%%%%% THEOREM Environments %%%%%%%%%% 					
% \newtheorem{conjecture}[theorem]{บทคาดการณ์}								
% \newtheorem{remark}[theorem]{หมายเหตุ}										
% \numberwithin{equation}{section}							
% \renewcommand\tablename{ตารางที่}
% \renewcommand\figurename{รูปที่}						
% \renewcommand{\bibname}{บรรณานุกรม}						
% \renewcommand{\indexname}{ดรรชนี}
\setbeamertemplate{caption}[numbered]	
\setbeamertemplate{theorems}[numbered]				
%%%%%%%%%%%%%%%%%%%%%%%%%%%%%%%%%%%%%%%%%%%%%%%

\mode<presentation>{
	\usetheme{Madrid}
    \usecolortheme[named=PineGreen]{structure}
}
\title[numerical algorithm for image inpainting]{A new numerical algorithm for TV-based image inpainting with its applications for restoring ancient Thai painting images and removing subtitles from animes}
\author[Pakkapon]{Pakkapon Phongthawee}
\institute[Silpakorn]{
 	Department of Mathematics\\
 	Silpakorn University \\}
\date[DPSTCON 2019]{DPST Student Conference on Science and Technology\\June 21-23,2019}
 
 \AtBeginSubsection[]{
 	\begin{frame}<beamer>
 		\frametitle{Outlines}
 		\tableofcontents [currentsection,currentsubsection]
     \end{frame}
}

\newcommand{\backupbegin}{
   \newcounter{finalframe}
   \setcounter{finalframe}{\value{framenumber}}
}
\newcommand{\backupend}{
   \setcounter{framenumber}{\value{finalframe}}
}


%\setbeamertemplate{item}[square]
%============================================================================
\begin{document}
    % \begin{frame} \titlepage \end{frame}
    \begin{frame}
        \begin{figure}[H]
            \centering
            \includegraphics[width=1\linewidth]{images/cover.png}
        \end{figure}
    \end{frame}
    \begin{frame}
        \frametitle{Image inpainting} 
        \begin{figure}[H]
            \centering
            \begin{subfigure}{0.3\linewidth}
                \centering
                \includegraphics[width=0.8\linewidth]{images/grayscale_inpaint/toinpaint.png}
                \caption{Damaged image}
            \end{subfigure}
            \begin{subfigure}{0.3\linewidth}
                \centering
                \includegraphics[width=0.8\linewidth]{images/grayscale_inpaint/inpaintdomain.png}
                \caption{Inpainted domain}
            \end{subfigure}
            \begin{subfigure}{0.3\linewidth}
                \centering
                \includegraphics[width=0.8\linewidth]{images/grayscale_inpaint/result_splitbergman.png}
                \caption{Restored image}
            \end{subfigure}
            \caption{Example of image inpainting}
        \end{figure}
    \end{frame}
    \begin{frame}
        \frametitle{Image inpainting problem}
        \begin{figure}[H]
            \centering
            \begin{subfigure}{0.4\linewidth}
                \centering
                \includegraphics[width=0.8\linewidth]{images/grayscale_inpaint/toinpaint.png}
                \caption*{{\large grayscale image}}
            \end{subfigure}
            \begin{subfigure}{0.4\linewidth}
                \centering
                \includegraphics[width=0.8\linewidth]{images/image_inpaint_synthetic/case02-toinpaint.png}
                \caption*{{\large color image}}
            \end{subfigure}
        \end{figure}
    \end{frame}
    \begin{frame}
        \frametitle{grayscale image}
        \begin{itemize}
            \item image domain $\Omega \subset \mathbb{R}^2$ 
            \item inpainting domain  $ D \subset \mathbb{R}^2$
            \item physical position $ \mathbf{x} = (x,y) \in \Omega $ 
            \item image intensity  $V \subset [0,\infty)$ 
            \item grayscale image $ u: \Omega \rightarrow V,\ z: \Omega \rightarrow V$
            \item without lose of generality  $ \Omega = [1,n]^2 $ and $ V = [0,1] $ which $n>0$ is positive integer 
        \end{itemize}
         \begin{figure}[h]
            \[
            \includegraphics[width=0.3\linewidth]{images/sample-domain.png}
            \]
            \caption{$D$ is an inpainting domain}
        \end{figure}
    \end{frame}    
    \begin{frame}
        \frametitle{Total variation model for grayscale image inpainting}
        \begin{align*}
        \min_{u} \{ \mathcal{J}(u) = \frac{1}{2} \int_{\Omega}\lambda (u-z)^2 d\Omega +  \int_{\Omega}  |\nabla u|  d\Omega \}
        \end{align*}
         \vspace{1cm}
        \begin{align*}
        \lambda=\lambda(\mathbf{x}) = \left \{ \begin{array}{ll}  \lambda_0, & x \in \Omega \textbackslash D \\ 0, & x \in D  \end{array} \right . 
        \end{align*}
        \let\thefootnote\relax\footnotetext{\tiny{T.F. Chan and J. Shen , “Mathematical models of local non-texture inpaintings”, SIAM Journal on Applied Mathematics, vol. 62, no. 3, pp. 1019–1043, 2001.}}	
    \end{frame}    
    \begin{frame}
        \frametitle{Numerical algorithm}
        \begin{itemize}
            \item[(1)] Explicit time marching
            \item[(2)] Fixed point iteration
            \item[(3)] Split Bregman
        \end{itemize}
    \end{frame}
    \begin{frame}
		\frametitle{Explicit time marching}
		\begin{align*}
		\min_{u} \{ \mathcal{J}(u) = \frac{1}{2} \int_{\Omega}\lambda (u-z)^2 d\Omega +  \int_{\Omega}  |\nabla u|  d\Omega \}
		\end{align*}
		$$ \Big \downarrow$$
		\begin{align*}
		\left \{ \begin{array}{ll}  - \nabla \cdot  \Big( \dfrac{\nabla u}{|\nabla u|} \Big) + \lambda (u-z) = 0,  & \hspace{1cm} \mathbf{x} \in (1,n)^2 \\ \dfrac{\partial u}{\partial \boldsymbol{n}} = 0, & \hspace{1cm} x \in \partial \Omega \end{array} \right .
        \end{align*}
        \let\thefootnote\relax\footnotetext{\tiny{L. I. Rudin, S. Osher, E. Fatemi, “Nonlinear total variation based noise removal algorithms", Physica D: Nonlinear Phenomena, vol 60, issues 1–4, pp. 259-268, 1992.}}						
    \end{frame} 
    \begin{frame}
        \frametitle{Explicit time marching}
        \begin{align*}
        u(\mathbf{x},t_{k+1})=u(\mathbf{x},t_{k})+\tau\left(\nabla \cdot\left(\dfrac{\nabla u (\mathbf{x},t_k)}{| \nabla u (\mathbf{x},t_k) | }\right) + \lambda(\mathbf{x})(u (\mathbf{x},t_k)-z(\mathbf{x})) \right)
        \end{align*}
        \begin{align*}
        u(\mathbf{x},t_0)=z \hspace{1cm} t_k=t_0+k\tau\ (\tau>0)  \hspace{1cm}  t_0=0
        \end{align*}
        \vspace{1cm}
        \begin{align*}
            u(\mathbf{x},t_0), u(\mathbf{x},t_1), u(\mathbf{x},t_2), u(\mathbf{x},t_3), ... ,  \textcolor{red}{u(\mathbf{x},t^{*})}
        \end{align*}
    \end{frame} 
    \begin{frame}
        \frametitle{Limitation of explicit time marching}
        \begin{align*}
        u(\mathbf{x},t_{k+1})=u(\mathbf{x},t_{k})+ \textcolor{red}{\tau}\left(\nabla \cdot\left(\dfrac{\nabla u (\mathbf{x},t_k)}{| \nabla u (\mathbf{x},t_k) | }\right) + \lambda(\mathbf{x})(u (\mathbf{x},t_k)-z(\mathbf{x})) \right)
        \end{align*}
        \vspace{1cm}
        \begin{align*}
            \textcolor{red}{\tau < 1}
        \end{align*}
    \end{frame} 
    \begin{frame}
        \frametitle{Fixed-point iteration}
        \begin{align*}
            - \nabla\cdot\left(\dfrac{\nabla u^{[\nu+1]}}{{| \nabla u |}^{[v]} }\right) + \lambda(u^{[\nu+1]}-z)  = 0,\ u^{[0]}=z
        \end{align*}
        \vspace{1cm}
        \begin{align*}
        u^{[0]}, u^{[1]}, u^{[2]}, u^{[3]}, ..., \textcolor{red}{u^{*}}    
        \end{align*}
        \let\thefootnote\relax\footnotetext{\tiny{C.R. Vogel and M.E. Oman,“Iterative methods for total variation denoising", SIAM Journal on Scientific Computing. vol. 17, pp. 227-238, 1996.}}
        %\note{นอกจากนี้ยังมีคณะวิจัยของ Vogel และ Oman ได้แนะนำวิธีการทำซ้ำแบบจุดตรึงสำหรับการกำจัดสัญญาณรบกวนไว้ ซึ่งสามารถนำมาประยุกต์กับวิธีการต่อเติมภาพ เริ่มจากการแนะนำดัชนีการทำซ้ำแบบจุดตรึง u=0,1,2,... และนิยามรูปแบบการทำซ้ำโดย }
    \end{frame} 
    \begin{frame}
        \frametitle{Numerical problem}
        \begin{figure}[H]
            \centering
            \includegraphics[width=0.2\linewidth]{images/grayscale_inpaint/result_splitbergman.png}
            \caption{Example of image which has numerical problem}
        \end{figure}
        \begin{align*}
            \tfrac{1}{| \nabla u |}=\tfrac{1}{\sqrt{u_x^2+u_y^2}} \rightarrow \infty
        \end{align*}
        \begin{align*}
            |\nabla u| \approx| \nabla u |_\beta=\sqrt{u_x^2+u_y^2+\beta},\ 0< \beta \ll 1
        \end{align*}
    \end{frame}
    \begin{frame}
        \frametitle{Split Bregman}
        \begin{align*}
            \min_{u} \{ \mathcal{J}(u) = \frac{1}{2} \int_{\Omega}\lambda (u-z)^2 d\Omega +  \int_{\Omega}  |\nabla u|  d\Omega \}
        \end{align*}
        $$ \Big \downarrow$$
		\begin{align*}
		    \min_{u,\boldsymbol{w}} \{ \mathcal{J}(u,\boldsymbol{w}) = \dfrac{1}{2} \int_{\Omega} \lambda(u-z)^2 d\Omega +  \int_{\Omega}  |\boldsymbol{w}| d\Omega \} \hspace{1cm}\text{ which } w = \nabla u
        \end{align*}
        $$ \Big \downarrow$$
		\begin{align*}
		    \min_{u,\boldsymbol{w}} \{ \mathcal{J}(u,\boldsymbol{w}) = \dfrac{1}{2} \int_{\Omega} \lambda(u-z)^2 d\Omega +  \int_{\Omega}  |\boldsymbol{w}|  d\Omega + \frac{\theta}{2} \int_{\Omega} (\boldsymbol{w} - \nabla u + \boldsymbol{b})^2 d\Omega \}
        \end{align*}
        \let\thefootnote\relax\footnotetext{\tiny{T. Goldstein and S. Osher,“The Split Bregman Method for L1-Regularized Problems", SIAM Journal on Imaging Sciences. vol. 2, issue 2, pp. 323-343, 2009.}}			
	\end{frame}  	
    \begin{frame}
        \frametitle{Split Bregman}
            \begin{align*}
        \min_{u,\boldsymbol{w}} \{ \mathcal{J}(u,\boldsymbol{w}) = \dfrac{1}{2} \int_{\Omega} \lambda(u-z)^2 d\Omega +  \int_{\Omega}  |\nabla \boldsymbol{w}|  d\Omega + \frac{\theta}{2} \int_{\Omega} (\boldsymbol{w} - \nabla u + \boldsymbol{b}) d\Omega \}
        \end{align*}
        $$ \Big \downarrow$$
        \begin{align*}
        u^{\text{New}}=\underset{u}{\arg\min} \{ \mathcal{J}_1(u) = \dfrac{1}{2} \int_{\Omega} \lambda(u-z)^2 d\Omega + \frac{\theta}{2} \int_{\Omega} (\boldsymbol{w}^{\text{old}} - \nabla u + \boldsymbol{b}^{\text{old}}) d\Omega \}
        \end{align*}
        \begin{align*}
        \boldsymbol{w}^{\text{New}}=\underset{\boldsymbol{w}}{\arg\min} \{ \mathcal{J}_2(\boldsymbol{w}) = \int_{\Omega}  |\nabla \boldsymbol{w}|  d\Omega  + \frac{\theta}{2} \int_{\Omega} (\boldsymbol{w} - \nabla u^{\text{New}} + \boldsymbol{b}^{\text{old}}) d\Omega \}
        \end{align*}
        \begin{align*}
        \boldsymbol{b}^{\text{New}}=\boldsymbol{b}^{\text{old}}+\nabla u^{\text{New}}-\boldsymbol{w}^{\text{New}}
        \end{align*}
        %\note{เราจะใช้วิธีการหาค่าต่ำที่สุดแบบสลับ โดยทำการตรึงค่า w จากนั้นเริ่มแก้ปัญหาย่อย แล้วนำ u ที่ได้มาตรึงค่า u เพื่อแก้ปัญหาย่อย w จากนั้นจึงทำการเปลี่ยนค่าตัวแปร b แล้วเริ่มทำซ้ำที่การหา u ใหม่อีกครั้ง จนกระทั่งค่านอร์มระหว่างรูป u ปัจจุบันและก่อนหน้าน้อยกว่าค่าที่กำหนด}
    \end{frame}  
    \begin{frame}
        \frametitle{Color image}
        \begin{figure}[H]
            \centering
            \includegraphics[width=0.4\linewidth]{images/rgb-space.png}
            \caption{Color image create from red,green and blue color\footnote{\tiny{ภาพจาก \url{https://commons.wikimedia.org/wiki/File:Additive_RGB_Circles-48bpp.png}  สืบค้นเมื่อวันที่ 25 กันยายน 2561}}}
        \end{figure}
    \end{frame}
    \begin{frame}
        \frametitle{Color image}
        \begin{align*}
             \boldsymbol{u},\ \boldsymbol{z} : \Omega  \rightarrow V
        \end{align*}
        \begin{align*}
            \Big \downarrow
        \end{align*}
        \begin{align*}
             \boldsymbol{u} = \begin{bmatrix} \textcolor{red}{u_1} \\ \textcolor{OliveGreen}{u_2} \\ \textcolor{blue}{u_3}   \end{bmatrix}, \ \boldsymbol{z} =\begin{bmatrix} \textcolor{red}{z_1} \\ \textcolor{OliveGreen}{z_2} \\ \textcolor{blue}{z_3} \end{bmatrix} : \Omega  \rightarrow V^3
        \end{align*} 
    \end{frame}
    \begin{frame}
        \frametitle{Total variation model for color image inpainting}
        \begin{align*}
        \min_{u} \{ \mathcal{J}(u) = \frac{1}{2} \int_{\Omega}\lambda (u-z)^2 d\Omega +  \int_{\Omega}  |\nabla u|  d\Omega \}
        \end{align*}
        \begin{align*}
        \Big \downarrow
        \end{align*}
        \begin{align*}
        \min_{u} \{ \mathcal{J}(u) = \underset{l=1}{\overset{3}{\sum}} 
        ( \frac{1}{2} \int_{\Omega}\lambda (u_l-z_l)^2 d\Omega +  \int_{\Omega}  |\nabla u_l|  d\Omega ) \}
        \end{align*}
    \end{frame}
    \begin{frame}
    \frametitle{Split Bregman for Color image}
        \begin{align*}
        \min_{u,\boldsymbol{w}} \{ \mathcal{J}(u,\boldsymbol{w}) = \dfrac{1}{2} \int_{\Omega} \lambda(u-z)^2 d\Omega +  \int_{\Omega}  |\nabla \boldsymbol{w}|  d\Omega + \frac{\theta}{2} \int_{\Omega} (\boldsymbol{w} - \nabla u + \boldsymbol{b}) d\Omega \}
        \end{align*}
        \begin{align*}
        \Big \downarrow
        \end{align*}
        \begin{align*}
        \min_{\boldsymbol{u},\boldsymbol{w_1},\boldsymbol{w_2},\boldsymbol{w_3}} \{ \mathcal{J}(\boldsymbol{u},\boldsymbol{w_1},\boldsymbol{w_2},\boldsymbol{w_3}) &= \underset{l=1}{\overset{3}{\sum}} (  \dfrac{1}{2} \int_{\Omega} \lambda(u_l-z_l)^2 d\Omega +  \int_{\Omega}  |\nabla \boldsymbol{w_l}|  d\Omega \\ &+ \frac{\theta}{2} \int_{\Omega} (\boldsymbol{w_l} - \nabla u_l+ \boldsymbol{b_l}) d\Omega ) \}
        \end{align*}
    \end{frame}
    \begin{frame}
        \frametitle{Quality measurement}
        \begin{itemize}
            \item[(1)] Peak Signal Noise Ratio (PSNR)
            \item[(2)] Structural Similarity (SSIM)
        \end{itemize}
    \end{frame}
    \begin{frame}
        \frametitle{Peak Signal Noise Ratio (PSNR)}
        \begin{align*}
        \text{PSNR}  = 10 \cdot log_{10} ( \frac{1}{\sqrt{\text{MSE}}} )  \hspace{1cm}
        \end{align*}
        \begin{itemize}
            \item[$\bullet$] MSE is mean square error which is MSE = $\frac{1}{nx \times ny} \sum (u - \tilde{u})^2 $
            \item[$\bullet$] $u$ is original image.
            \item[$\bullet$] $\tilde{u}$  is restored image from numerical method.
            \item[$\bullet$] The unit of PSNR is the decibel (dB).
        \end{itemize}
    \end{frame}
    \begin{frame}
        \frametitle{Structural Similarity (SSIM)}
        \begin{align*}
        \text{SSIM}(u,\tilde{u}) = \frac{(2\mu_u\mu_{\tilde{u}} + 0.0001)(2\sigma_{u\tilde{u}} + 0.0009)}{(\mu_u^2+\mu_{\tilde{u}}^2+0.0001)(\sigma_u^2+\sigma_{\tilde{u}}^2+0.0009)}
        \end{align*}
        \begin{itemize}
            \item[$\bullet$] $u$ is original image.
            \item[$\bullet$] $\tilde{u}$ is restored image from numerical method.
            \item[$\bullet$] $\mu_u$ is average of $u$
            \item[$\bullet$] $\mu_{\tilde{u}}$ is average of $\tilde{u}$
            \item[$\bullet$]  $\sigma_u$ is variation of $u$ 
            \item[$\bullet$] $\sigma_{\tilde{u}}$ is vriation of $\tilde{u}$
            \item[$\bullet$] SSIM has range between 0 to 1.
        \end{itemize}
    \end{frame}
    \begin{frame}
        \frametitle{Synthetic image}
        \begin{figure}[H]
            \centering
            \begin{subfigure}{0.15\linewidth}
                \centering
                \includegraphics[width=0.9\linewidth]{images/image_inpaint_synthetic/case01-original.png}
            \end{subfigure}
            \begin{subfigure}{0.15\linewidth}
                \centering
                \includegraphics[width=0.9\linewidth]{images/image_inpaint_synthetic/case02-original.png}
            \end{subfigure}
            \begin{subfigure}{0.15\linewidth}
                \centering
                \includegraphics[width=0.9\linewidth]{images/image_inpaint_synthetic/case03-original.png}	
            \end{subfigure}
            \begin{subfigure}{0.15\linewidth}
                \centering
                \includegraphics[width=0.9\linewidth]{images/image_inpaint_synthetic/case04-original.png}
            \end{subfigure}
            \begin{subfigure}{0.15\linewidth}
                \centering
                \includegraphics[width=0.9\linewidth]{images/image_inpaint_synthetic/case05-original.png}	
            \end{subfigure}
            \caption{Original image}
        \end{figure}
        \begin{figure}[H]
            \centering
            \begin{subfigure}{0.15\linewidth}
                \centering
                \includegraphics[width=0.9\linewidth]{images/image_inpaint_synthetic/case01-toinpaint.png}
            \end{subfigure}
            \begin{subfigure}{0.15\linewidth}
                \centering
                \includegraphics[width=0.9\linewidth]{images/image_inpaint_synthetic/case02-toinpaint.png}
            \end{subfigure}
            \begin{subfigure}{0.15\linewidth}
                \centering
                \includegraphics[width=0.9\linewidth]{images/image_inpaint_synthetic/case03-toinpaint.png}
            \end{subfigure}
            \begin{subfigure}{0.15\linewidth}
                \centering
                \includegraphics[width=0.9\linewidth]{images/image_inpaint_synthetic/case04-toinpaint.png}
            \end{subfigure}
            \begin{subfigure}{0.15\linewidth}
                \centering
                \includegraphics[width=0.9\linewidth]{images/image_inpaint_synthetic/case05-toinpaint.png}
            \end{subfigure}
            \caption{Damaged image}
        \end{figure}
        \begin{align*}
            \text{iteration} \leq 10,000 \text{ round } \hspace{1cm}
            \frac{|| u_{new} - u_{old} ||}{|| u_{new} ||} \geq 10^{-4}
        \end{align*}
    \end{frame}
    \begin{frame}
        \frametitle{Restoration result}
        \begin{figure}[H]
            \centering
            \begin{subfigure}{0.15\linewidth}
                \centering
                \includegraphics[width=0.9\linewidth]{images/result_ex1/timemarch01.png}
            \end{subfigure}
            \begin{subfigure}{0.15\linewidth}
                \centering
                \includegraphics[width=0.9\linewidth]{images/result_ex1/timemarch02.png}
            \end{subfigure}
            \begin{subfigure}{0.15\linewidth}
                \centering
                \includegraphics[width=0.9\linewidth]{images/result_ex1/timemarch03.png}
            \end{subfigure}
            \begin{subfigure}{0.15\linewidth}
                \centering
                \includegraphics[width=0.9\linewidth]{images/result_ex1/timemarch04.png}
            \end{subfigure}
            \begin{subfigure}{0.15\linewidth}
                \centering
                \includegraphics[width=0.9\linewidth]{images/result_ex1/timemarch05.png}
            \end{subfigure}
            \caption{Explicit time marching}
        \end{figure}
        \begin{figure}[H]
            \centering
            \begin{subfigure}{0.15\linewidth}
                \centering
                \includegraphics[width=0.9\linewidth]{images/result_ex1/fixpoint01.png}
            \end{subfigure}
            \begin{subfigure}{0.15\linewidth}
                \centering
                \includegraphics[width=0.9\linewidth]{images/result_ex1/fixpoint02.png}
            \end{subfigure}
            \begin{subfigure}{0.15\linewidth}
                \centering
                \includegraphics[width=0.9\linewidth]{images/result_ex1/fixpoint03.png}
            \end{subfigure}
            \begin{subfigure}{0.15\linewidth}
                \centering
                \includegraphics[width=0.9\linewidth]{images/result_ex1/fixpoint04.png}
            \end{subfigure}
            \begin{subfigure}{0.15\linewidth}
                \centering
                \includegraphics[width=0.9\linewidth]{images/result_ex1/fixpoint05.png}
            \end{subfigure}
            \caption{Fixed point iteration }
        \end{figure}
        \begin{figure}[H]
            \centering
            \begin{subfigure}{0.15\linewidth}
                \centering
                \includegraphics[width=0.9\linewidth]{images/result_ex1/splitbergman01.png}
            \end{subfigure}
            \begin{subfigure}{0.15\linewidth}
                \centering
                \includegraphics[width=0.9\linewidth]{images/result_ex1/splitbergman02.png}
            \end{subfigure}
            \begin{subfigure}{0.15\linewidth}
                \centering
                \includegraphics[width=0.9\linewidth]{images/result_ex1/splitbergman03.png}
            \end{subfigure}
            \begin{subfigure}{0.15\linewidth}
                \centering
                \includegraphics[width=0.9\linewidth]{images/result_ex1/splitbergman04.png}
            \end{subfigure}
            \begin{subfigure}{0.15\linewidth}
                \centering
                \includegraphics[width=0.9\linewidth]{images/result_ex1/splitbergman05.png}
            \end{subfigure}
            \caption{Split Bregman}
        \end{figure}
    \end{frame}
    \begin{frame}
        \frametitle{Performance of numerical method}
        \begin{table}[H]
        \centering
        \captionsetup{justification=centering}
            \begin{tabular}[ht]{|l|c|c|c|c|}
                \hline
                Method  & Processed time  (Second) & PSNR (dB) & SSIM \\
                \hline
                Explicit time marching & 120.68 & 16.72 & 0.9960 \\
                Fixed point iteration & 74.81 & 38.67 & 0.9999 \\
                \textcolor{red}{Split Bregman} & \textcolor{red}{14.06} & \textcolor{red}{39.42} & \textcolor{red}{0.9999}  \\
                \hline
            \end{tabular}
        \caption{Average image restoration result of numerical method  \\  $\lambda = 250, \beta = 10^{-5}, \tau = 10^{-5}, \theta = 5 $}
        \end{table}	
    \end{frame}
    \begin{frame}
        \frametitle{Intial solution}
        \begin{figure}[H]
            \centering
            \begin{subfigure}{0.8\linewidth}
                \centering
                \includegraphics[width=1\linewidth]{images/image_inpaint_synthetic/image_inital_solution.png}
            \end{subfigure}
            \caption{Find intial solution by using image pyramid}
        \end{figure}
    \end{frame}
    \begin{frame}
        \frametitle{Image pyramid}
        \begin{figure}[H]
            \centering
            \begin{subfigure}{0.6\linewidth}
                \centering
                \includegraphics[width=0.8\linewidth]{images/image_inpaint_synthetic/image_pyramid.png}
            \end{subfigure}
            \caption{Image pyramid method}
        \end{figure}
        \let\thefootnote\relax\footnotetext{\tiny{	E.H. Andelson and C.H. Anderson and J.R. Bergen and P.J. Burt and J.M. Ogden. "Pyramid methods in image processing". 1984}}
    \end{frame}
    \begin{frame}
        \frametitle{Restoration result with image pyramid}
        \begin{table}[H]
            \centering
            \begin{tabular}[ht]{|l|c|c|c|c|}
                \hline
                Iteration step  & Processed time  (Second) & PSNR (dB) & SSIM \\
                \hline
                Without Image pyramid & 17.38 & 39.42 & 0.9999 \\
                10/1/1/10000 & 13.52 & 39.38 & 0.9999 \\
                10/3/3/10000 & 11.86 & 39.54 & 0.9999 \\
                10/10/10/10000 & 9.26 & 40.17 & 0.9999\\
                100/1/1/10000 & 10.28 & 39.04 & 0.9999\\
                100/3/3/10000 & 10.28 & 39.80 & 0.9999\\
                100/10/10/10000 & 9.27 & 40.12 & 0.9999 \\
                \hline
            \end{tabular}
            \caption{Average result of Split Bregman method with Image pyramid}
        \end{table}	
    \end{frame}
    \begin{frame}
        \frametitle{Iteration on Finest level}
        \begin{figure}[H]
            \centering
            \begin{subfigure}{0.4\linewidth}
                \centering
                \includegraphics[width=0.6\linewidth]{images/just10enough/only5time.png}
                \caption{5 times}
            \end{subfigure}
            \begin{subfigure}{0.4\linewidth}
                \centering
                \includegraphics[width=0.6\linewidth]{images/just10enough/only10time.png}
                \caption{10 times}
            \end{subfigure}
            \begin{subfigure}{0.4\linewidth}
                \centering
                \includegraphics[width=0.6\linewidth]{images/just10enough/only50time.png}			
                \caption{50 times}
            \end{subfigure}
            \begin{subfigure}{0.4\linewidth}
                \centering
                \includegraphics[width=0.6\linewidth]{images/just10enough/only100time.png}			
                \caption{100 times}
            \end{subfigure}
            \caption{{\footnotesize image pyramid with iteration 10/10/10 but have difference iteration on finest step}}
        \end{figure}
    \end{frame}
    \begin{frame}
        \frametitle{Restoration result with only 10 iteration on finest step}
        \begin{table}[H]
            \centering
            \begin{tabular}[ht]{|l|c|c|c|c|}
                \hline
                Iteration step  & Processed time  (Second) & PSNR (dB) & SSIM \\
                \hline
                Without Image pyramid & 0.37 & 17.26 & 0.9963  \\
                10/1/1/10 & 0.40 & 28.54 & 0.9993 \\
                \textcolor{red}{10/3/3/10} & \textcolor{red}{0.33} & \textcolor{red}{29.83}  & \textcolor{red}{0.9994} \\
                10/10/10/10 & 0.38 & 32.56 & 0.9995 \\
                100/1/1/10 & 0.34 & 31.50 & 0.9999 \\
                100/3/3/10 & 0.36 & 31.99 & 0.9999 \\
                100/10/10/10 & 0.38 & 33.39 & 0.9998 \\
                \hline
            \end{tabular}
            \caption{{\small Average result of Resotration with only 10 itearation on finest step}}
        \end{table}	
    \end{frame}
    \begin{frame}
		\frametitle{Algorithm for Thai painting restoration}
		\begin{figure}[H]
			\centering
			\begin{subfigure}{0.7\linewidth}
				\centering
				\includegraphics[width=1\linewidth]{images/method_thaiart/step_thaiart.png}
			\end{subfigure}
			\caption{Our algorithm for Thai painting restoration}
        \end{figure}
	\end{frame}
	\begin{frame}
		\frametitle{Algorithm for Thai painting restoration}
		\begin{figure}[H]
			\centering
			\begin{subfigure}{0.5\linewidth}
				\centering
				\includegraphics[width=1\linewidth]{images/method_thaiart/flowchart_thaiart.png}
			\end{subfigure}
			\caption{Our algorithm's flow chart}
		\end{figure}
	\end{frame}
    \begin{frame}
        \frametitle{Thai painting restoration}
        \begin{figure}[H]
            \centering
            \begin{subfigure}{0.15\linewidth}
                \centering
                \includegraphics[width=0.9\linewidth]{images/thaiart/case01-original.png}
            \end{subfigure}
            \begin{subfigure}{0.15\linewidth}
                \centering
                \includegraphics[width=0.9\linewidth]{images/thaiart/case02-original.png}
            \end{subfigure}
            \begin{subfigure}{0.15\linewidth}
                \centering
                \includegraphics[width=0.9\linewidth]{images/thaiart/case03-original.png}
            \end{subfigure}		
            \begin{subfigure}{0.15\linewidth}
                \centering
                \includegraphics[width=0.9\linewidth]{images/thaiart/case04-original.png}
            \end{subfigure}
            \begin{subfigure}{0.15\linewidth}
                \centering
                \includegraphics[width=0.9\linewidth]{images/thaiart/case05-original.png}
            \end{subfigure}
            \caption{Original Thai painting}
        \end{figure}
        \begin{figure}[H]
            \centering
            \begin{subfigure}{0.15\linewidth}
                \centering
                \includegraphics[width=0.9\linewidth]{images/thaiart/case01-toinpaint.png}
            \end{subfigure}
            \begin{subfigure}{0.15\linewidth}
                \centering
                \includegraphics[width=0.9\linewidth]{images/thaiart/case02-toinpaint.png}
            \end{subfigure}
            \begin{subfigure}{0.15\linewidth}
                \centering
                \includegraphics[width=0.9\linewidth]{images/thaiart/case03-toinpaint.png}			
            \end{subfigure}
            \begin{subfigure}{0.15\linewidth}
                \centering
                \includegraphics[width=0.9\linewidth]{images/thaiart/case04-toinpaint.png}			
            \end{subfigure}
            \begin{subfigure}{0.15\linewidth}
                \centering
                \includegraphics[width=0.9\linewidth]{images/thaiart/case05-toinpaint.png}			
            \end{subfigure}
            \caption{Damage Thai painting}
        \end{figure}
    \end{frame}
    \begin{frame}
        \frametitle{Thai painting restoration result}
        \begin{figure}[H]
            \centering
            \begin{subfigure}{0.15\linewidth}
                \centering
                \includegraphics[width=0.9\linewidth]{images/result_ex4/splitbergman_case01.png}
            \end{subfigure}
            \begin{subfigure}{0.15\linewidth}
                \centering
                \includegraphics[width=0.9\linewidth]{images/result_ex4/splitbergman_case02.png}
            \end{subfigure}
            \begin{subfigure}{0.15\linewidth}
                \centering
                \includegraphics[width=0.9\linewidth]{images/result_ex4/splitbergman_case03.png}			
            \end{subfigure}
            \begin{subfigure}{0.15\linewidth}
                \centering
                \includegraphics[width=0.9\linewidth]{images/result_ex4/splitbergman_case04.png}			
            \end{subfigure}
            \begin{subfigure}{0.15\linewidth}
                \centering
                \includegraphics[width=0.9\linewidth]{images/result_ex4/splitbergman_case05.png}			
            \end{subfigure}
            \caption{Split Breman}
        \end{figure}
        \begin{figure}[H]
            \centering
            \begin{subfigure}{0.15\linewidth}
                \centering
                \includegraphics[width=0.9\linewidth]{images/result_ex4/multisplitbergman_case01.png}
            \end{subfigure}
            \begin{subfigure}{0.15\linewidth}
                \centering
                \includegraphics[width=0.9\linewidth]{images/result_ex4/multisplitbergman_case02.png}
            \end{subfigure}
            \begin{subfigure}{0.15\linewidth}
                \centering
                \includegraphics[width=0.9\linewidth]{images/result_ex4/multisplitbergman_case03.png}			
            \end{subfigure}
            \begin{subfigure}{0.15\linewidth}
                \centering
                \includegraphics[width=0.9\linewidth]{images/result_ex4/multisplitbergman_case04.png}			
            \end{subfigure}
            \begin{subfigure}{0.15\linewidth}
                \centering
                \includegraphics[width=0.9\linewidth]{images/result_ex4/multisplitbergman_case05.png}			
            \end{subfigure}
            \caption{Our algorithm}
        \end{figure}
    \end{frame}
    \begin{frame}
        \frametitle{Comparison between 2 algorithms}
        \begin{figure}[H]
            \centering
            \begin{subfigure}{0.4\linewidth}
                \centering
                \includegraphics[width=0.6\linewidth]{images/result_ex4_scaleup/original.png}
                \caption{Original image}
            \end{subfigure}
            \begin{subfigure}{0.4\linewidth}
                \centering
                \includegraphics[width=0.6\linewidth]{images/result_ex4_scaleup/toinpaint.png}
                \caption{Damaged image}
            \end{subfigure}
            \begin{subfigure}{0.4\linewidth}
                \centering
                \includegraphics[width=0.6\linewidth]{images/result_ex4_scaleup/splitbregman.png}			
                \caption{Split Bremgan}
            \end{subfigure}
            \begin{subfigure}{0.4\linewidth}
                \centering
                \includegraphics[width=0.6\linewidth]{images/result_ex4_scaleup/multisplitbregman.png}	
                \caption{Our algorithm}
            \end{subfigure}
            \caption{Comparison of images that have been expanded 4 times}
        \end{figure}
    \end{frame}
    \begin{frame}
        \frametitle{Thai painting restoration result}
        \begin{table}[H]
            \centering
            \begin{tabular}[ht]{|l|c|c|c|c|}
                \hline
                Method  & Processing time  (Second) & PSNR (dB) & SSIM \\
                \hline
                Split Bregman & 2.72 & 34.89 & 1.0000 \\ 
                \textcolor{red}{Our algorithm} & \textcolor{red}{0.39} & \textcolor{red}{35.30} & \textcolor{red}{1.0000} \\
                \hline
            \end{tabular}
            \caption{Average Thai painting restoration}
        \end{table}	
    \end{frame}
    \begin{frame}
        \frametitle{Anime subtitle removal}
        \begin{figure}[H]
            \centering
            \includegraphics[width=0.8\linewidth]{images/inori-preview.png}
            \caption{ Festival Asia Special Video - feat. Inori Aizawa }
        \end{figure}
    \end{frame}
    \begin{frame}
        \frametitle{Lorem Ipsum}
            \begin{figure}[H]
            \centering
            \includegraphics[width=0.85\linewidth]{images/lorem4lang.png}
            \caption{Lorem Ipsum in 4 languages \footnote{ \tiny{Lorem ipsum in difference language https://en.wikipedia.org/wiki/Lorem\_ipsum,  https://ja.wikipedia.org/wiki/Lorem\_ipsum and  https://th.wikipedia.org/wiki/ลอเร็มอิปซัม, 25 November 2018} } }
        \end{figure}
    \end{frame}
    \begin{frame}
        \frametitle{Split video}
            \begin{figure}[H]
            \centering
            \includegraphics[width=0.8\linewidth]{images/inori-subbed-preview.png}
            \caption{Split video into 5 parts}
        \end{figure}
    \end{frame}
    \begin{frame}
    \frametitle{Video and image}
        \begin{figure}[H]
            \centering
            \includegraphics[width=0.6\linewidth]{images/anime-frame.png}
            \caption{Video is sequences of image}
        \end{figure}
    \end{frame}
    \begin{frame}
        \frametitle{Finding subtitle}
        \begin{figure}[H]
            \centering
            \includegraphics[width=0.8\linewidth]{images/subtitle-remove/thisisanimesubtitle.png}
            \caption{Anime subtitle always has black border}
        \end{figure}
    \end{frame}
    \begin{frame}
        \frametitle{Finding subtitle}
        \begin{figure}[H]
            \begin{subfigure}{0.4\linewidth}
                \centering
                \includegraphics[width=0.8\linewidth]{images/subtitle_detection/detection-original.png}
                \caption{Anime with subtitle}
            \end{subfigure}
            \begin{subfigure}{0.4\linewidth}
                \centering
                \includegraphics[width=0.8\linewidth]{images/subtitle_detection/detection-threshold.png}
                \caption{mark black as white}
            \end{subfigure}
            \begin{subfigure}{0.4\linewidth}
                \centering
                \includegraphics[width=0.8\linewidth]{images/subtitle_detection/detection-inverse.png}
                \caption{inverse color}
            \end{subfigure}
            \begin{subfigure}{0.4\linewidth}
                \centering
                \includegraphics[width=0.8\linewidth]{images/subtitle_detection/detection-blackfill.png}
                \caption{remove white which attach to edge}
            \end{subfigure}
            \begin{subfigure}{0.4\linewidth}
                \centering
                \includegraphics[width=0.8\linewidth]{images/subtitle_detection/detection-erode-opening.png}
                \caption{remove smaller or larger object}
            \end{subfigure}
            \begin{subfigure}{0.4\linewidth}
                \centering
                \includegraphics[width=0.8\linewidth]{images/subtitle_detection/detection-stoke.png}
                \caption{expand inpaint domain}
            \end{subfigure}
            \caption{our method to finding subtitle}
        \end{figure}
    \end{frame}
    \begin{frame}
        \frametitle{Finding subtitle result}
        \begin{table}[H]
            \centering
            \footnotesize
            \begin{tabular}[ht]{|l|c|c|c|c|}
                \hline
                Language  & Pixel in inpaint domain & Pixel which detected & Pixel which error & Error (percent) \\
                \hline
                Thai & 23,222,220 & 24,083,125 & 2,141,201 & 9.22 \\
                English & 27,278,745 & 28,598,424 & 3,714,321 & 13.62 \\
                Japanese & 28,544,173 & 30,103,466 & 3,740,971 & 13.11 \\
                \hline
            \end{tabular}
            \caption{Average error of finding inpainting domain in various language subtitles}
        \end{table}	
    \end{frame}
    \begin{frame}
        \frametitle{Sequence of image and intial solution}
        \begin{figure}[H]
            \centering
            \includegraphics[width=0.8\linewidth]{images/skipborrow/frame_sequence.png}
            \caption{Image sequence in video which has subtitle}
        \end{figure}
    \end{frame}
    \begin{frame}
        \frametitle{Consider similarities with SSIM}
        \begin{figure}[H]
            \centering
            \begin{subfigure}{0.4\linewidth}
                \centering
                \includegraphics[width=0.95\linewidth]{images/skipborrow/prevframe.png}
                \caption{Previous frame}
            \end{subfigure}
            \begin{subfigure}{0.4\linewidth}
                \centering
                \includegraphics[width=0.95\linewidth]{images/skipborrow/currentframe.png}
                \caption{Current frame}
            \end{subfigure}
            \bigskip
            \begin{subfigure}{0.4\linewidth}
                \centering
                \includegraphics[width=0.95\linewidth]{images/skipborrow/prevframeinverse.png}
                \caption{Area to find SSIM in previous frame}
            \end{subfigure}
            \begin{subfigure}{0.4\linewidth}
                \centering
                \includegraphics[width=0.95\linewidth]{images/skipborrow/currentframeinverse.png}
                \caption{Area to find SSIM in current frame}
            \end{subfigure}
            \caption{Area to calculate SSIM for skip frame and borrow frame method}
        \end{figure}
    \end{frame}
    \begin{frame}
        \frametitle{Skip frame}  
        \begin{figure}[H]
            \centering
            \includegraphics[width=0.6\linewidth]{images/skipborrow/flowchart-skip.png}
            \caption{Skip frame method}
        \end{figure}      
    \end{frame}
    \begin{frame}
        \frametitle{Borrow frame}      
        \begin{figure}[H]
            \centering
            \includegraphics[width=0.6\linewidth]{images/skipborrow/flowchart-borrow.png}
            \caption{Borrow frame method}
        \end{figure}   
    \end{frame}
    \begin{frame}
        \frametitle{Skip and borrow frame}  
        \begin{figure}[H]
            \centering
            \includegraphics[width=0.6\linewidth]{images/skipborrow/flowchart-skipandborrow.png}
            \caption{Skip and borrow frame method}
        \end{figure}        
    \end{frame}
    \begin{frame}
        \frametitle{Skip/Borrow frame result}
        \begin{table}[H]
            \centering
            
            \begin{tabular}[ht]{|l|c|c|c|c|}
                \hline
                Method  & Processing time  (Second) & PSNR (dB) & SSIM \\
                \hline
                Our first algorithm & 141.29 & 31.39  &  0.9510\\
                Skip frame & 89.29 & 29.07 & 0.9408 \\
                Borrow frame & 132.78 & 32.20 & 0.9655\\
                \textcolor{red}{Skip and borrow frame} & \textcolor{red}{75.76} & \textcolor{red}{29.33} & \textcolor{red}{0.9454} \\
                \hline
            \end{tabular}
            \caption{Average result of removing subtitle from anime}
        \end{table}	
    \end{frame}
    \begin{frame}
        \frametitle{Algorithm for removing subtitle from anime}  
        \begin{figure}[H]
            \centering
            \includegraphics[width=0.6\linewidth]{images/skipborrow/flowchart-skipandborrow.png}
            \caption{Skip and borrow frame method}
        \end{figure}        
    \end{frame}
    \begin{frame}
        \frametitle{Removing subtitle result}
        \begin{figure}[H]
            \centering
            \begin{subfigure}{0.45\linewidth}
                \centering
                \includegraphics[width=0.97\linewidth]{images/subtitle-remove/beforesubtitleremove.png}
                \caption{Before}
            \end{subfigure}
            \begin{subfigure}{0.45\linewidth}
                \centering
                \includegraphics[width=0.95\linewidth]{images/subtitle-remove/aftersubtitleremove.png}
                \caption{After}
            \end{subfigure}
            \caption{sample frame from anime}
        \end{figure}
    \end{frame}
    \begin{frame}
        \frametitle{Removing subtitle result}
        \begin{table}[H]
            \centering
            \captionsetup{justification=centering}
            \begin{tabular}[ht]{|l|c|c|c|c|}
                \hline
                Method  & Processed time  (Second) & PSNR (dB) & SSIM \\
                \hline
                Split Bregman & 5073.08 & 32.88 & 0.9654 \\
                \textcolor{red}{Our algorithm} & \textcolor{red}{75.76} & \textcolor{red}{29.33} & \textcolor{red}{0.9454} \\
                \hline
            \end{tabular}
            \caption{
                Average result of removing subtitle with Split Bregman and our algorithm }
        \end{table}	
    \end{frame}
    \begin{frame}
        \frametitle{Thai painting restoration}
        \centering
        \begin{figure}[H]
            \centering
            \includegraphics[width=0.7\linewidth]{images/colab_sample.png}
        \end{figure}
        \begin{figure}[H]
            \centering
            \includegraphics[width=0.1\linewidth]{images/colab_qr.png}
            \caption{Try by yourself at https://bit.ly/demothai}
        \end{figure}
    \end{frame}
    \begin{frame}
		\frametitle{Program for removing subtitle from anime}
		\begin{figure}
			\includegraphics[width=0.8\linewidth]{images/demo_anime/file.png}
			\caption{file for removing subtitle (https://bit.ly/demo-anime-inpaint)}
		\end{figure}
		\begin{figure}
			\includegraphics[width=0.8\linewidth]{images/demo_anime/notepad.png}
			\caption{source code in SubtitleRemove.avs}
		\end{figure}
	\end{frame}
    \begin{frame}
        \centering
        \Huge{Thank you}
    \end{frame}
    \backupbegin
    \begin{frame}
        \frametitle{Inpainting is Ill-posed problem}
        \begin{figure}[H]
            \centering
            \begin{subfigure}{0.45\linewidth}
                \centering
                \includegraphics[width=0.90\linewidth]{images/q_and_a/meme_elephant.png}
                \caption{Inpainting Problem}
            \end{subfigure}
            \begin{subfigure}{0.45\linewidth}
                \centering
                \includegraphics[width=0.90\linewidth]{images/q_and_a/meme_elephant_answer.png}
                \caption{Solution of the problem.}
            \end{subfigure}
            \caption{Red area can be elephant leg and lion.\footnote{Baa dum tsss
             https://9gag.com/gag/aer4VwB, 10 March 2562}}
        \end{figure}
    \end{frame}
    \begin{frame}
        \frametitle{ROF-model for denoising and TV-based model for inpainting}
        \begin{align*}
            \min_{u} \{ \mathcal{J}(u) = \frac{1}{2} \int_{\Omega}\lambda (u-z)^2 d\Omega +  \int_{\Omega}  |\nabla u|  d\Omega \}
            \end{align*}
             \vspace{1cm}
             \begin{columns}
                \begin{column}{0.5\textwidth}
                    \begin{align*}
                        \textcolor{red}{\lambda=\lambda_0}
                    \end{align*}
                    \begin{center}
                        \textcolor{red}{Denosing}
                    \end{center}
                \end{column}
                \begin{column}{0.5\textwidth}  %%<--- here
                    \begin{align*}
                        \textcolor{blue}{\lambda=\lambda(\mathbf{x}) = \left \{ \begin{array}{ll}  \lambda_0, & x \in \Omega \textbackslash D \\ 0, & x \in D  \end{array} \right . }
                    \end{align*}
                    \begin{center}
                        \textcolor{blue}{Inpainting}
                    \end{center}
                \end{column}
            \end{columns}            
            \let\thefootnote\relax\footnotetext{\tiny{
                Pascal Getreuer, Rudin-Osher-Fatemi Total Variation Denoising using Split Bregman, Image Processing On Line, 2 (2012), pp. 74–95.}}	
    \end{frame}
    \begin{frame}
        \frametitle{Why we need to remove subtitle}
        \begin{figure}
			\includegraphics[width=0.8\linewidth]{images/q_and_a/theleader.png}
			\caption{ Ling Feng Zhe has Chinese subtitle because it had produced in China}
		\end{figure}
    \end{frame}
    \begin{frame}
        \frametitle{If we don't remove subtitle}
        \begin{figure}[H]
            \centering
            \begin{subfigure}{0.45\linewidth}
                \centering
                \includegraphics[width=0.90\linewidth]{images/q_and_a/paopunjin.jpg}
                \caption{Justice Pao (Ch. 3)}
            \end{subfigure}
            \begin{subfigure}{0.45\linewidth}
                \centering
                \includegraphics[width=0.90\linewidth]{images/q_and_a/sample_of_subtitle_replace.png}
                \caption{Wise and Foolish (Thai PBS)}
            \end{subfigure}
            \begin{subfigure}{0.45\linewidth}
                \centering
                \includegraphics[width=0.90\linewidth]{images/q_and_a/nhk_subtitle.png}
                \caption{Nuclear ghost town (NHK World)}
            \end{subfigure}
            \caption{How TV channel work on forgien language subtitle}
        \end{figure}
    \end{frame}
    \backupend
\end{document}